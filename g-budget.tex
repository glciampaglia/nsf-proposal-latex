\section*{Budget Justification}

\subsection*{Personnel}

\textbf{Giovanni Luca Ciampaglia, Ph.D.} Principal Investigator (Assistant Professor, University of South Florida) will lead the project and be primarily responsible for the research related to analysis of Facebook URL Shares dataset, the development of algorithms for news recommendation to diverse audiences, analysis of robustness of diversity-based recommender systems. Duties include research design, analysis, disseminating results of the research, and filing reports. He will mentor one graduate student at the University of South Florida (RA1), as well as any undergraduate researchers hired as part of REU supplements. Salary support is requested for 0.70 summer month of each year, with a 3\% increase each year, for a total of \$15,474.

\textbf{Research Assistant}. This proposal will support one graduate student as Research Assistants in years 1 and 2. The assistant (RA1) will be part of the Ph.D. program in Computer Science \& Engineering at the University of South Florida, and will work under the supervision of PI Ciampaglia. The graduate students will be primarily responsible for developing algorithms for recommending news to a diverse audience, and for testing its robustness against shilling attacks via simulation and empirical analysis. Support for RA1 will be during the calendar year, for a total of \$52,000.

% {\textbf{Undergraduate Students} This project will support undergraduate students in some research aspects, developing documentation, software tutorials, etc. For this, \$7,500 in year one, with a 3\% increase each year. This amount works out to 2 student, \$10 per hour, 10 hours per week, for 37.5 weeks of the year (i.e., approximately the academic year). The students staffing this role are expected to change every 6 to 12 months. Recruitment will follow the guideline in Section~\ref{sec:broader:bpc}.}

\subsection*{Fringe Benefits}

Fringe benefits for each award are calculated according to guidelines set by the University of South Florida. Fringe for faculty including retirement is 18\% plus \$1,473 for the cost of health insurance. Fringe for graduate students is 0.2\% plus the cost of health insurance. The USF health insurance annual coverage 2019-20 for an RA is \$2,371 (\$890 for Fall, and \$1,481 for Spring/Summer). 

\subsection*{Graduate Student Tuition}

Graduate student tuition rates for RA1 are calculated according to the guidelines set by the University of South Florida. For RA1, \$39,010 is requested to support up to 18 credit hours per graduate student per year. USF tuition rates are \$431.43 for In-State and \$877.17 for Out-of-State.

\subsection*{Materials and Supplies}

\$1,850 is requested for miscellaneous supplies over the life of the project. These include the workstations, software, tools, and reference materials necessary to complete the proposed work.

\subsection*{Travel}

The funds will be used to support the PI, graduate student, and any undergraduate students (from REU supplement) for travel expenses, broken down as:
\begin{enumerate}
    \item  \$1,500 per year is requested for 2 domestic conferences; \item \$5,000 per year is requested for travel expenses of project personnel and of two unfunded collaborators --- Dr. Filippo Menczer (Indiana University) and Alessandro Flammini (Indiana University) --- for trips to Tampa, Fla. to visit PI Ciampaglia at University of South Florida, and to Ann Arbor, Mi. to visit Co-PI Nyhan at University of Michigan.   
\end{enumerate}

Travel reimbursements will be in accordance with Florida Statutes 112.061.

\subsection{Direct and Indirect Costs}

The University of South Florida has negotiated an indirect cost rate of 49.5\% for federal grants (source: \url{http://www.usf.edu/research-innovation/sr/documents/indirect-rates.pdf}). Indirect is not charged on Permanent Equipment or Graduate Student Tuition. 

Total Direct Costs of the project are \$105,472. Total Indirect Costs: \$44,528.

The total amount of this request is \$150,000.

