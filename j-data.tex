\section*{Data Management Plan}

The main contribution of this research is the development of new methods for news recommendation in social media newsfeeds that prioritizes consumption from diverse (i.e., bipartisan) audiences.
%
These research outcomes will be disseminated through peer-reviewed technical publications, source code releases, and benchmark data. The emphasis of data management will be on faithful and
reproducible record keeping, with an emphasis on transparency and accountability in methods utilized.

\subsection*{Data storage and dissemination}

Because of the interdisciplinary nature of this collaboration, results will be disseminated to multiple audiences, with a priority for journal/conference venues that cater to interdisciplinary audiences in the social sciences and in computer science. In particular, we will target archival Computer Science conferences such as RecSys, CHI, CSCW, ICWSM, WWW, KDD, and CIKM; multidisciplinary journals such as \emph{Nature}, \emph{Science}, \emph{PNAS}, \emph{Nature Communications}, \emph{Nature Human Behavior}, \emph{Science Advances}, \emph{Scientific Reports}, \emph{EPJ Data Science}, \emph{Proc. Roy. Soc.: Interface}; non-archival interdisciplinary conferences like NetSci, CCS (Conference on Complex Systems), IC2S2 (International Conference on Network Science); and, finally, Political Science journals such as \emph{American Journal of Political Science}, \emph{American Politics Research}, \emph{Public Choice}, \emph{Political Research Quarterly}. 

In addition to appearing on publisher's websites, preprint publications will be posted on both a project website and arXiv. Publications will be produced and stored on a commercial LaTeX host, such as Overleaf.

Source code will be developed and stored on commercial git systems, such as Atlassian Bitbucket or GitHub, in private repositories. Once results for techniques are published, source code will be open-sourced under the GNU GPL or equivalent license.

All project data will be stored on servers at University of South Florida and/or Indiana University.

\paragraph{Commercial Applications} For commercial applications of our technology, we will adhere to the USF policy on intellectual property (source: \url{http://www.usf.edu/research-innovation/about-usfri/policies.aspx}). Under this policy, we will work through USF's Patents and Licensing division.

\paragraph{Human Study Data} 

All human study data will be secured in accordance with University's IRB requirements. This data will be anonymized and released on the project website in accordance with IRB requirements. For any data obtained from Social Science One, LLC, we will follow any additional requirements from Social Science One and its institutional partners (such as the Social Science Research Council and Facebook, Inc.).
% Submission for IRB review is currently underway (\textit{IRB protocol \#Pro00038180}).

\subsection*{Data Retention}

All data generated under this project will be maintained for the life of the project. At the conclusion of the project, data related to major findings and software developed will be archived at the University of South Florida for a period of no less that 5 years. Source code stored externally and archival publication preserves those records indefinitely.

\subsection*{Data Protection}

Data will be stored on one or servers that offer RAID protection for data, preserving the data in the case of any single hard disk failure. If more than one disk is lost, we rely on cloud-based data backups to Google Drive and Box. Data is constantly synchronized meaning that in the case of total failure, only very recent changes may be lost. For source code, the commercial services we plan to use (i.e., Overleaf, Bitbucket, GitHub, etc.) offer high-levels of reliability for long-term storage. 

\subsection*{Data Privacy}

Data access is primarily controlled via IP restrictions and Access Control technologies. For further protective siloing, only members of this project will have access to sensitive project materials. Commercial storage services offer similar access control.