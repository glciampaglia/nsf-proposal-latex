\section*{Collaboration Plan}

The project team spans three different institutions (University of South Florida, University of Michigan, and Indiana University) and various disciplines --- computer science (Ciampaglia and Menczer), political science (Nyhan), Physics (Flammini and Menczer), and network science (Ciampaglia, Flammini, Menczer). Ciampaglia, Menczer, and Flammini have collaborated together in the past, and they have considerable experience bridging computer science, physics, and network science in their work. Even though Ciampaglia recently joined a new institution, they keep in touch on a regular basis as part of ongoing projects. It is the first time that the three of them collaborate with Nyhan, bridging a new discipline (political science). Therefore, to guarantee success at such an early stage of a new interdisciplinary collaboration, it is important to imbue the activities of the team with some structure regarding coordination and communication. 

PI Ciampaglia will lead the project; he will mentor one graduate student at USF for the full duration of the project, and will be responsible for coordinating the three institutions. Co-PI Nyhan will mentor on graduate student and one undergraduate researcher for part of the project. The two groups, together with collaborators Flammini and Menczer, will communicate on a regular basis via teleconferencing software (Zoom or Skype). These meetings will be the primary venue of communication of the joint collaboration. During them, research assistants will give updates and present preliminary results to the rest of the team. Goals and activities for the next period will be discussed and agreed upon by the whole team in a collegial way.

The use of additional channels of communication (e.g. Slack, Google Groups) will be evaluated by the team as the need arise, but in general electronic communications outside of meetings will be kept at a minimum: Email will be used for scheduling meetings and to share brief updates and external materials; it will not be used for in-depth conversations. Github issues (or a suitable issue tracking system) will be used to keep a record of conversations about software development, especially at the stage when software will be released to the public and will attract external users.

In addition to teleconference meetings, the team will also meet in person. Funds have been requested to support travel expenses of Flammini and Menczer from Indiana University to University of South Florida, where Ciampaglia will host them for short visits during the summer. The same funds may also be used to support the travel of either Ciampaglia and/or the USF research assistant to University of Michigan, to meet with Co-PI Nyhan and his team there. Regardless of the destination, these short visits will be used as `sprints', for example to finalize manuscripts or to let team members work in close collaboration on software development and/or data analysis.